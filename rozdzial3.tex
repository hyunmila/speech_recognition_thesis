\chapter{Panel HMI}
\label{}

Głównym celem, do którego miało za zadanie prowadzić wybranie odpowiedniego algorytmu uczenia maszynowego do rozpoznawania głosowego było wykonanie panelu HMI do sterowania głosem. HMI (ang. Human-Machine Interface) jest to z definicji panel służący do połączenia człowieka z urządzeniem lub oprogramowaniem. Pomimo, że pojęcie to określa każdy ekran umożliwiający interakcję z maszyną, jest głównie używane w kontekście przemysłowym do komunikacji ze sterownikami PLC (ang. Programmable Logic Controllers). 
Panel HMI maszyny może być używany między innymi do:
\begin{itemize}
    \item sterowania,
    \item monitorowania wejść i wyjść,
    \item wizualizacji danych,
    \item śledzenia procesów.
\end{itemize}
Panele takie często są budowane do wytrzymywania w ekstremalnych warunkach w zakładach przemysłowych i fabrykach i przystosowywane do działania na odległość. Ich powstanie jest powiązane głównie z potrzebą optymalizacji - dzięki urządzeniom monitorującym procesy zdalnie, pracownicy zakładów zamiast marnować czas na sprawdzanie każdej maszyny, mogą to zrobić w jednym, dedykowanym do tego miejscu. 

%---------------------------------------------------------------------------

\section{Środowisko pracy}
\label{}

Stanowiskiem pracy jest urządzenie z systemem operacyjnym Ubuntu 20.04. W celach użycia narzędzia \textit{Gazebo} \cite{gazebo} służącego do symulowania systemów w czasie rzeczywistym, zainstalowano środowisko \textit{Robot Operating System} (skr.: ROS) \cite{ros}. 

ROS \cite{ros} jest to otwarte oprogramowanie posiadające zestaw bibliotek i narzędzi umożliwiający budowanie aplikacji dla robotów, kompatybilne z językiem programowania \textit{Python}. Dystrybucja zainstalowana na urządzeniu to \textit{ROS Noetic Ninjemys} \cite{noetic}, odmiana dedykowana dla używanego systemu operacyjnego z długoterminowym wsparciem ze strony producenta. 

Do zobrazowania działania algorytmu w symulacji został użyty model TurtleBot3. 

Diagram ...

%---------------------------------------------------------------------------

\section{Aplikacja}
\label{}

CTK
Sterowanie może przebiegać za pomocą ...