\chapter{Podsumowanie}
\label{cha:podsumowanie}

% W pracy zostało zawarte zestawienie oraz ewaluacja podejść, 
W pracy przedstawiono przegląd metod, jakie można wykorzystać w automatycznym rozpoznawaniu mowy. Zestawienie algorytmów pozwoliło na weryfikację założeń projektowych oraz wstępną ewaluację, gdzie spośród trzech zaproponowanych rozwiązań zostało wybrane jedno, sugerujące największą dokładność w wykonywaniu powierzonego zadania. Użyta w ewaluacji metryka jest jednym z głównych i najbardziej wiarygodnym wskaźnikiem algorytmów służących do rozpoznawania lub tłumaczenia mowy.  

Wykonany na potrzeby weryfikacji rozwiązania panel HMI umożliwia sterowanie robotem mobilnym zarówno w sposób manualny jak i głosowy, zapewniając podstawowe komendy sprawdzające użyteczność zaproponowanego rozwiązania. Narzędzia użyte do zbudowania środowiska, w którym możliwe było sprawdzenie jego działania to aktualnie jedne z najlepszych dostępnych na rynku otwartych oprogramowań, które są używane nie tylko w prywatnych projektach, ale sprawdzają się również edukacyjnie i komercyjnie. Budowa samego panelu HMI wymagała podejścia iteracyjnego, gdzie po napisaniu każdej z części, konieczne było testowanie jego działania i spójności z algorytmem. Zaproponowane rozwiązanie wykonania panelu za pomocą biblioteki Pythona było jednocześnie łatwe do zrealizowania jak i skuteczne w swoim zastosowaniu. Biblioteka zapewniła przy wykonanym nakładzie pracy możliwość prostej weryfikacji funkcjonalności i rezultatów.

Największe wyzwanie w projekcie stanowiła ewaluacja algorytmów uczenia maszynowego. Wykonanie tego zadania wymagało dogłębnego zrozumienia na każdym poziomie działania sieci neuronowych oraz procesu ich uczenia się na danych. Przeprowadzenie ewaluacji wymagało sporego nakładu pracy w postaci przeszukiwania i analizy zestawów danych celem znalezienia odpowiedniego, spełniającego wymagania sprzętowe, środowiskowe i wydajnościowe. Niemniej jednak, proces ten zaowocował w późniejszym etapie zadowalającymi wynikami, które mogły zostać faktycznie użyte do sterowania robotem w symulacji. 

Przeprowadzona w pracy analiza i zaproponowane rozwiązania są jedynie małym krokiem w stronę zastosowania algorytmów uczenia maszynowego w praktyce życia codziennego. Dziedzina ta ciągle się rozwija, pisane są nowe modele i zbierane nowe zestawy danych, które jeszcze lepiej będą mogły radzić sobie ze stawianymi przed nimi zadaniami optymalizacji i usprawniania otaczającego nas świata. Przedstawiona propozycja sterowanego mową interfejsu człowiek-maszyna, w przyszłości ma możliwość rozbudowy o więcej funkcjonalności lub zmodyfikowania na potrzeby innych zadań. Przed nami stoi tylko decyzja: co zrobimy dalej?
